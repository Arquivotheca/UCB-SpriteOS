\documentstyle{article}

\begin{document}

\begin{center}
{\large {\bf NACHOS}}

% \vspace{.2in}
% Tom Anderson\\
% Wayne Christopher\\
% Steven Procter

% \vspace{.2in}
% Computer Science 162\\
% January 29, 1992

\vspace{.5in}
Assignment \#1: Build a thread system\\
Due date: Monday, Feb. 17, 5:00 p.m.
\end{center}

\vspace{.2in}

In this assignment, we give you part of a working thread system; 
your job is to complete it, and then to use it to solve
several synchronization problems.  

The first step is to read and understand the partial thread system 
we have written for you.  This thread system implements thread fork, 
thread completion, along with semaphores for synchronization.  
Run the program `nachos' for a simple test of our code.
Trace the execution path (by hand) for the simple test case
we provide.

When you trace the execution path, it is helpful to keep track
of the state of each thread and which procedures are on each thread's 
execution stack.
You will notice that when one thread calls SWITCH, another thread
starts running, and the first thing the new thread does is 
to return from SWITCH.
We realize this comment will seem cryptic to you at this point, but you
will understand threads once you understand
why the SWITCH that gets called is different from the SWITCH that returns.
(Note: because gdb does not understand threads, you will get bizarre 
results if you try to trace in gdb across a call to SWITCH.)

The files for this assignment are:

\begin{description}

\item main.cc, test1.cc --- a simple test of our thread routines.

\item thread.h, thread.cc --- thread data structures and
thread operations such as thread fork, thread sleep and thread finish.

\item scheduler.h, scheduler.cc --- manages the list of threads that
are ready to run.

\item synch.h, synch.cc --- synchronization routines: semaphores, locks, 
and condition variables.

\item list.h, list.cc --- generic list management (LISP in C++).

\item synchlist.h, synchlist.cc --- synchronized access to lists using
locks and condition variables (useful as an example of the use 
of synchronization primitives).

\item system.h, system.cc --- Nachos startup/shutdown routines.

\item utility.h, utility.cc --- some useful definitions and debugging routines.

\item switch.h, switch.s --- assembly language magic for starting 
up threads and context switching between them.

\item interrupt.h, interrupt.cc --- manage enabling and disabling
interrupts as part of the machine emulation.

\item timer.h, timer.cc --- emulate a clock that periodically causes 
an interrupt to occur.

\item stats.h -- collect interesting statistics.

\end{description}

For this assignment, all groups must do every part.  We suggest that 
each person in a group code all of this assignment separately, since 
we will ask you to do similar problems on the exams, and you will
need the practice.  There should be no busy-waiting in any of your 
solutions.  Each part has equal weight in the grading.

Properly synchronized code should work no matter what order the 
scheduler chooses to run the threads on the ready list.  In other 
words, we should be able to put a call to Thread::Yield (causing the scheduler
to choose another thread to run) anywhere in your code where interrupts 
are disabled without changing the correctness of your code.   
You will be asked to write properly synchronized code as part of the 
later assignments, so understanding how to do this is crucial to
being able to do the project.

To aid you in this, code linked in with Nachos will cause Thread::Yield 
to be called on your behalf in a repeatable but unpredictable way.
Nachos code is repeatable in that if you call it repeatedly with the 
same arguments, it will do exactly the same thing each time.
However, if you invoke ``nachos -rs \#'', with a different number each
time, calls to Thread::Yield will be inserted at different places in the code.

Make sure to run various test cases against your solutions to 
these problems; for instance, for part two, create multiple producers
and consumers and demonstrate that the output can vary, within certain 
boundaries.  Where the question calls for an explanation, include
that in the README.

Warning: in our implementation of threads, each thread is assigned a 
small, fixed-size execution stack.  This may cause bizarre problems 
(such as segmentation faults at strange lines of code) if you declare 
large data structures to be automatic variables (e.g., ``int buf[1000];'').
You will probably not notice this during the semester, but if you do,
you may change the size of the stack by modifying the StackSize define in 
switch.h.

\begin{description}

\item{1.}
Implement locks and condition variables using semaphores as the only
synchronization primitive.  We have provided the public interface to these 
objects in synch.h.  You need to define the private data and implement
the interface.

\item{2.}
Implement producer/consumer communication through a bounded buffer,  
using locks and condition variables.  (A solution to the bounded buffer 
problem is described in Silberschatz, Peterson and Galvin, using 
semaphores.)

The producer places characters from the string "Hello world" into the 
buffer one character at a time; it must wait if the buffer is full.
The consumer pulls characters out of the buffer one at a time
and prints them to the screen; it must wait if the buffer is empty.
Test your solution with a buffer that holds more than one
character at a time and with multiple producers and consumers.

\item{3.} The local laundromat has just entered the computer age.
As each customer enters, he or she puts coins into
slots at one of two stations and types in
the number of washing machines he/she will need.  The stations
are connected to a central computer that
automatically assigns available machines and outputs
tokens that identify the machines to be used.  The
customer puts laundry into the machines and inserts each
token into the machine indicated on the token.  When a machine finishes
its cycle, it informs the computer that it is available again.
The computer maintains an array {\em available[NMACHINES]} whose elements are
non-zero if the corresponding machines are available (NMACHINES is
a constant indicating how many machines there are in the
laundromat), and a semaphore {\em nfree} that indicates how many 
machines are available.

The code to allocate and release machines is as follows:

\begin{verbatim}
int allocate()	/* Returns index of available machine.*/
{
  int i;

  P(nfree);	/* Wait until a machine is available */
  for (i=0; i < NMACHINES; i++)
    if (available[i] != 0) {
      available[i] = 0;
      return i;
    }
}

release(int machine)	/* Releases machine */
{
  available[machine] = 1;
  V(nfree);
}
\end{verbatim}

The {\em available} array is initialized to all ones, and {\em nfree} is
initialized to NMACHINES.

\begin{description}

\item{(a)} It seems that
if two people make requests at the two stations at the same time, they
will occasionally be assigned the same machine.  This has resulted in
several brawls in the laundromat, and you have been called in by the
owner to fix the problem.  Assume that one thread handles each
customer station.  Explain how the same washing machine can be assigned
to two different customers.

\item{(b)} Modify the code to eliminate the problem.

\item{(c)} Re-write the code to solve the synchronization problem 
using locks and condition variables instead of semaphores.

\end{description}

\item{4.} You've just been hired by Mother Nature to help her out with the
chemical reaction to form water, which she doesn't seem to be
able to get right due to synchronization problems.  The trick is to
get two H atoms and one O atom all together at the same time.  The
atoms are threads.  Each H atom invokes a procedure {\em hReady} when it's
ready to react, and each O atom invokes a procedure {\em oReady}
when it's ready.  For this problem, you are to write the code for
{\em hReady} and {\em oReady}.  The procedures must delay until there
are at least two H atoms and one O atom present, and then one
of the procedures must call the procedure {\em makeWater} (which
just prints out a debug message that water was made).
After the {\em makeWater} call, two 
instances of {\em hReady} and one instance of {\em oReady} should return.
Write the code for {\em hReady} and {\em oReady} using
either semaphores or locks and condition variables for synchronization.
Your solution must avoid starvation and busy-waiting.

\item{5.} You have been hired by Caltrans to synchronize traffic over a
narrow light-duty bridge on a public highway.  Traffic may only cross
the bridge in one direction at a time, and if there are ever more than
3 vehicles on the bridge at one time, it will collapse  under their
weight.  In this system, each car is represented by one thread,
which executes the procedure {\em OneVehicle} when it arrives
at the bridge:

\begin{verbatim}
OneVehicle(int direc)
{
  ArriveBridge(direc);
  CrossBridge(direc);
  ExitBridge(direc);
}
\end{verbatim}

In the code above, {\em direc} is either 0 or 1;  it gives the
direction in which the vehicle will cross the bridge.

\begin{description}

\item{(a)} Write the procedures {\em ArriveBridge}
and {\em ExitBridge} (the {\em CrossBridge} procedure should just print 
out a debug message), using locks and condition variables.  {\em ArriveBridge} 
must not return until it safe
for the car to cross the bridge in the given direction (it
must guarantee that there will be no head-on collisions or
bridge collapses).  {\em ExitBridge} is called to indicate
that the caller has finished crossing the bridge;  {\em ExitBridge}
should take steps to let additional cars cross the bridge.
This is a lightly-travelled rural bridge, so
you do not need to guarantee fairness or freedom from starvation.

\item{(b)} In your solution, if a car arrives while traffic is currently moving
in its direction of travel across the bridge, but there is another
car already waiting to cross in the opposite direction, will the
new arrival cross {\em before} the car waiting on the other side,
{\em after} the car on the other side, or is it impossible to
say?  Explain briefly.

\end{description}

\end{description}

\end{document}

\item{6.}
Implement (non-preemptive) priority scheduling.  

Modify the thread scheduler to always return the 
highest priority thread.  (You will need to create a new constructor 
for Thread to take another parameter -- the priority level of the thread;
leave the old constructor alone since we'll need it for backward
compatibility.)  You may assume that there are a fixed, small 
number of priority levels -- for this assignment, you'll only need two levels.

Can changing the relative priorities of producers and consumers
have any affect on the output?  For instance, what happens with 
two producers and one consumer, when one of the
producers is higher priority than the other?  What if the two
producers are at the same priority, but the consumer is at high
priority?

\item{7.} You have been hired to simulate one of the Cal fraternities.
Your job is to write a computer program to pair up men and women
as they enter a Friday night mixer.  Each man and each woman will
be represented by one thread.  When the man or woman enters the
mixer, its thread will call one of two procedures, \fIman\fR
or \fIwoman\fR, depending on the sex of the thread.  You must
write C code to implement these procedures.  Each procedure takes
a single parameter, \fIname\fR, which is just an integer name for
the thread.  The procedure must wait until there is an available
thread of the opposite sex, and must then exchange names
with this thread.
Each procedure must return the integer name of the thread
it paired up with.  Men and women may enter the fraternity
in any order, and many threads may call the \fIman\fR and
\fIwoman\fR procedures simultaneously.  It doesn't matter
which man is paired up with which woman (Cal frats aren't
very choosy), as long as each pair contains one man and one woman
and each gets the other's name.  Use semaphores and shared
variables to implement the two procedures.  Be sure to
give initial values for the semaphores and indicate which
variables are shared between the threads.  There must
not be any busy-waiting in your solution.
