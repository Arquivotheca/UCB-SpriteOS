\documentstyle{article}
	
\begin{document}

\begin{center}
{\large {\bf NACHOS}}

% \vspace{.2in}
% Tom Anderson\\
% Wayne Christopher\\
% Steven Procter
% \vspace{.2in}
% Computer Science 162\\
% April 1, 1992

\vspace{.5in}
Assignment \#4: Virtual Memory \\
Due date: Thursday, April 2, 5:00 p.m. \\
\end{center}

\vspace{.2in}

The fourth phase of Nachos is to support virtual memory.
We provide no new code for this assignment; your job is
to modify your implementation for assignment 3 to provide
the abstraction of an (almost) unlimited virtual memory size
to each user program.  

The illusion of unlimited memory is provided by the operating system
by using main memory as a cache for the disk.
Page tables were used in assignment 3 to simplify memory 
allocation and to isolate failures from one address space from 
affecting other programs.  For this assignment, page tables 
serve a few more purposes: the valid bit in the page table entry tells
the hardware which virtual pages are in memory and which are stored on disk,
and the hardware sets the use bit in the page table entry whenever
a page is referenced and the dirty bit whenever the page is modified.

When a program references a page that is not resident in memory
(detected by when the page table entry is not valid), the hardware
generates a {\em page fault} exception, trapping to the kernel.
The operating system kernel then reads the page in from disk,
sets the page table entry to point to the new page, and then resumes 
the execution of the user program.  Of course, the kernel must  
first find space in memory for the incoming page, potentially
writing some other page back to disk, if it has been modified.

As with any caching system, performance depends on the policy
used to decide which things are kept in memory and which
are only stored on disk.  
On a page fault, the kernel must decide which page to replace;
ideally, it will throw out a page that will not referenced for 
a long time, keeping pages in memory those that are soon to be 
referenced.  Another consideration is that if the replaced page 
has been modified, the page must be first saved to disk before the needed
page can be brought in; many virtual memory systems (such as UNIX)
avoid this extra overhead by writing modified pages to disk in 
advance, so that any subsequent page faults can be completed more quickly.

While there are many similarities between virtual memory and disk
block caching, there are some important differences.
Program references and file access patterns display both temporal 
locality (references are likely to be to locations referenced
in the recent past) and spatial locality (referenced locations are likely
to be near other recently referenced locations).  However, while
files are commonly read or written sequentially, program
references are less easy to characterize, since instructions, data
and the stack all have non-sequential access pattern.
Thus, while read-ahead may be a good idea for optimizing a file cache,
it would probably work less well for a virtual memory system.

There are two parts to this assignment; each is weighted equally.
If your file system is not working, you may use UNIX instead, but
20\% of your grade will be based on whether you use
your file system.  

\begin{description}
\item{1.}
Implement virtual memory.  For this, you will need routines
to move a page from disk to memory and from memory to disk.
We recommend that you use a Nachos file as backing store.
In order to find
unreferenced pages to throw out on page faults, you will need
to keep track of all of the pages in the system which are currently
in use.  A simple way to do this is to keep a ``core map'', which
is basically a reverse page table -- instead of translating virtual
page numbers to physical pages, a core map translates physical page numbers
to the virtual pages that are stored there.

\item{2.}
Optimize the performance of your virtual memory system
in running the ``perftest'' user program.  This program 
exercises the virtual memory system by multiplying
two large matrices together.

For this assignment, you may alter the test program to use
a different implementation of matrix multiply, if that helps 
performance, provided that it still correctly multiplies the matrices 
together!

One further twist: the combination of the disk cache and the main
memory size must be no more than 48 pages (or disk sectors).  
Currently, the main memory size is 32 pages; you may increase this
if you do not have a disk cache.  As in assignment 2, space taken
by meta-data such as open file descriptors, the bitmap, etc. must be 
added to the disk cache size, if they are stored permanently in memory.
Berkeley Sprite, by the way, keeps a common pool of memory that is shared by
the virtual memory system and by the file system.  One way of doing
this in Nachos is to use Machine::mainMemory as the space for the file cache.

\end{description}

\end{document}
